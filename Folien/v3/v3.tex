\documentclass[11pt]{beamer}
\usepackage{lmodern}
\usepackage{graphicx}
\usepackage{hyperref}
\usetheme{Singapore}
\author{Gerd Graßhoff}
\title{Algorithmische Geschichte und Philosophie der Wissenschaften, Vorl 4}
\date{9. Mai 2019}

\begin{document}
	\begin{frame}[plain]
		\maketitle
	\end{frame}
	

\section{Erkenntnisgegenstand}

	\begin{frame}
	\frametitle{Wissenschaftstheoretische Schlüsselbegriffe}
	\begin{itemize}
		\item "Detect"
		\item Gegenstand: Modell eines Objekts
		\item Messungrozess / Instrument
		\item kausales Differenzargument
	\end{itemize}
\end{frame}

	\begin{frame}
	\frametitle{Entdeckungsaussagen}
	\begin{itemize}
		\item Daten aus ''abstracts''
	\end{itemize}
\end{frame}

	\begin{frame}
	\frametitle{Satzanalyse}
	\begin{itemize}
		\item \href{https://explosion.ai/demos/}{Demos}
		\item \href{https://spacy-vis.apps.allenai.org/spacy-parser}{parser}
	\end{itemize}
\end{frame}


	\begin{frame}
	\frametitle{Satzanalyse: Bericht ja / nein}
	\begin{itemize}
		\item sentences[339]
		\item 439: We detected these systems in the F110W and F160W filters through our reanalysis of archival Hubble Space Telescope (HST)
	\end{itemize}
\end{frame}



\end{document}